%====================================================================
\chapter{Installation}
%====================================================================


\section{Introduction}
%--------------------------------------------------------------------

This chapter explains how \application{C-BGP} can be built from
sources. The build process of \application{C-BGP} has been carefully
generated using the GNU \application{autoconf}, \application{automake}
and \application{libtool} utilities. There are great chances that it
will work on most UNIX operating systems with a working C compilation
chain. The build process has been tested on various UNIX systems,
mainly Linux and FreeBSD workstations as well as some Solaris
systems. From time to time, the build process is also tested under
Microsoft Windows using the \application{Cygwin} library.

However, if the build process fails on your system, don't be
discouraged and contact the developpers on the sourceforge
mailing-list \verb|c-bgp-users@lists.sourceforge.net|


\section{Requirements}
%--------------------------------------------------------------------

There are a few libraries and tools required by
\application{C-BGP}. These libraries and tools must be installed on
your system before you attempt to build \application{C-BGP}.

\begin{itemize}
\item The GNU readline library, also known as
  \application{libreadline}, is required to provide a user-friendly
  command-line interface including auto-completion and input
  history. The GNU readline library is freely available from
  \url{http://www.gnu.org}. It is probably also available as a package
  on your system. Note that the library and its development
  headers are required. The GNU readline library is not mandatory as
  \application{C-BGP} contains a minimal replacement. It is however
  highly recommended.

\item The library of {\it Perl Compatible Regular Expressions} (PCRE),
  also known as the \application{libpcre} library is required to allow
  AS-Path based routing filters. This library is freely available from
  \url{http://www.pcre.org}. It is probably also available as a
  package on your system. Note that the library and its development
  headers are required. This library is mandatory.

\item The \application{zlib} compression library is required to allow
  reading compressed routing tables. This library is available from
  \url{http://www.zlib.net/}. It is probably also available as a
  package on your system. Note that the library and its development
  headers are required. This library is mandatory.

\item The \application{libbz2} compression library is recommended to
  allow reading compressed routing tables. This library is available
  from \url{http://bzip.org}. Is is probably also available as a
  package on your system. Note that the library and its development
  headers are required. This library is not mandatory.

\item The \application{libgds} library is a library of generic data
  structures developped for \application{C-BGP}. It is available from
  Sourceforge at
  \url{http://libgds.sourceforge.net}. Section~\ref{sec:inst-libgds}
  explains how to setup and install this library. This library is
  mandatory.

\item The \application{pkg-config} utility is required to allow
  \application{C-BGP} to link with the \application{libgds}
  library. The \application{pkg-config} utility can be obtained from
  \url{http://pkg-config.freedesktop.org/wiki/}. It is probably also
  available as a package on your system.

\end{itemize}

\section{Installation of the \application{libgds} library}
\label{sec:inst-libgds}
%--------------------------------------------------------------------

This section describes how to build the \application{libgds}
library. This library is required by \application{C-BGP} as it
provides most of \application{C-BGP}'s data structures. The
\application{libgds} library is freely available from Sourceforge at
\url{http://libgds.sourceforge.net}. 

To install \application{libgds}, the following steps need to be
followed. First, the latest source archive must be downloaded. The
source archive has a name such as \filename{libgds-x.y.z.tar.gz} where
\filename{x.y.z} denotes the library version. This archive must be
uncompressed in a temporary directory. This should create a directory
named \filename{libgds-x.y.z}. This directory must be entered, then
the \filename{configure} script must be executed. This script will
check that everything is available on your system to build
\application{libgds}. This script should run without any problem. When
the execution of \filename{configure} is finished, the build process
can be launched with \filename{make clean} followed by
\filename{make}. That will actually build the library.

\begin{Verbatim}[commandchars=\\\{\}]
\prompt{bash\$} \command{cd /tmp}
\prompt{bash\$} \command{tar xvzf libgds-x.y.z.tar.gz}
\prompt{bash\$} \command{cd libgds-x.y.z}
\prompt{bash\$} \command{./configure}
\prompt{bash\$} \command{make clean}
\prompt{bash\$} \command{make}
\end{Verbatim}

Once the build process is finished, the library must be
installed using \filename{make install}. The default installation
prefix is \filename{/usr/local}. This means that to install, root
privileges are required. To proceed with the installation, it is
needed to login as root user and type \filename{make install} or to use
\filename{sudo make install}. The installation process will install the
library file under \filename{<prefix>/lib} and the library development
headers under \filename{<prefix>/include/libgds}.

\begin{Verbatim}[commandchars=\\\{\}]
\prompt{bash\$} \command{sudo make install}
Password:
\end{Verbatim}

For those users that do not have root privileges, there is a mean to
change the installation prefix. Section~\ref{sec:inst-options}
discusses this process at length.

Once the library is installed it might be needed to check that the
linker can find it. This is a system dependent task. Under the Linux
operating system, there is usually a file named
\filename{/etc/ld.so.conf} that lists the directories where the linker
must search for libraries. The directory \filename{/usr/local/lib}
should be added to this file if it is not yet there. Then,
\filename{/sbin/ldconfig} must be run in order to update the linker's
database. Running \filename{ldconfig} requires root privilege.

Finally, the installation parameters of \application{libgds} are
stored in a file under \filename{/usr/local/lib/pkgconfig}. It is
needed to inform the \command{pkg-config} utility of this
location. This can be done by adding this path to the
\envar{PKG\_CONFIG\_PATH} environment variable, as follows:

\begin{Verbatim}[commandchars=+\{\}]
+prompt{bash$} +command{export PKG_CONFIG_PATH= \}
  +command{$PKG_CONFIG_PATH:/usr/local/lib/pkgconfig}
\end{Verbatim}


\section{Installation of \application{C-BGP}}
%--------------------------------------------------------------------

This section describes hot to build \application{C-BGP} once every
dependency (including \application{libgds}) have been properly
installed. The installation process is fairly similar to that of
\application{libgds}.

First, the sources archive \filename{cbgp-x.y.z.tar.gz} must be
downloaded from Sourceforge at \url{http://c-bgp.sourceforge.net}. The
sources archive has a name such as \filename{cbgp-x.y.z.tar.gz} where
\filename{x.y.z} denotes the version. This archive must be
uncompressed in a temporary directory. This will create a new
directory named \filename{cbgp-x.y.z}. This directory must be entered,
then, the \filename{configure} script must be run. Once the
\filename{configure} script has finished, the build process can be
launched using \filename{make clean} followed by \filename{make}.

\begin{Verbatim}[commandchars=\\\{\}]
\prompt{bash\$} \command{tar xvzf cbgp-x.y.z.tar.gz}
\prompt{bash\$} \command{cd cbgp-x.y.z}
\prompt{bash\$} \command{./configure}
\prompt{bash\$} \command{make clean}
\prompt{bash\$} \command{make}
\prompt{bash\$} \command{make install}
\end{Verbatim}

You can check that \application{C-BGP} is correctly built and
installed by issuing the following command. This command should
provide the version of \application{C-BGP} and list the supported
options.

\begin{Verbatim}[commandchars=\\\{\}]
\prompt{bash\$} \command{cbgp -h}
C-BGP Routing Solver version 2.0.0-rc3

  Copyright (C) 2002-2011 Bruno Quoitin
  Networking Lab
  Computer Science Institute
  University of Mons (UMONS)
  Mons, Belgium
  
  This software is free for personal and educational uses.
  See file COPYING for details.


Usage: cbgp [OPTIONS]

  -h             display this message.
  -l LOGFILE     output log to LOGFILE instead of stderr.
  -c SCRIPT      load and execute SCRIPT file.
                 (without this option, commands are taken from stdin)
  -e COMMAND     execute the given command
  -D param=value defines a parameter
\prompt{bash\$}
\end{Verbatim}

\section{Installation in another directory}
\label{sec:inst-other-dir}
%--------------------------------------------------------------------

In certain cases, it is needed to install the \application{libgds}
library and \application{C-BGP} under a non-default, non-standard,
directory. For example, this is needed when the user has not enough
privileges to copy files under the default install directory. This
section explains how this can be done.

When configuring \application{libgds} and \application{C-BGP}, an
additional option \option{-{}-prefix}=\parameter{PATH} must be
provided to the \filename{configure} script. This option will change
the default installation prefix and use the provided \parameter{PATH}
instead. For instance, in order to install under directory
\filename{/home/user/projects}, the following commands would be
needed:

\begin{Verbatim}[commandchars=+\{\}]
+prompt{bash\$} +command{./configure --prefix=/home/user/projects}
+prompt{bash\$} +command{make clean}
+prompt{bash\$} +command{make}
+prompt{bash\$} +command{make install}
+prompt{bash\$} +command{export PKG_CONFIG_PATH= \}
  +command{$PKG_CONFIG_PATH:/home/user/projects/lib/pkgconfig}
\end{Verbatim}

To install \application{C-BGP} under \filename{/home/user/projects},
the same \option{-{}-prefix} option must be used, as shown in the following example:

\begin{Verbatim}[commandchars=\\\{\}]
\prompt{bash\$} \command{./configure --prefix=/home/user/projects}
\prompt{bash\$} \command{make}
\prompt{bash\$} \command{make install}
\end{Verbatim}

The final step consists in informing the linker of the location where
\application{libgds} library and some of \application{C-BGP}'s
internal libraries were installed. The exact details of this operation
depend on the operating system. Under Linux, there is a file named
\filename{/etc/ld.so.config} which lists all the directories where the
linker is expected to find libraries. You can add
\filename{/home/user/projects/lib} to this file. Another approach is
to rely on the \envar{LD\_LIBRARY\_PATH} environment variable.
Please refer to the documentation of your operating system.

\begin{Verbatim}[commandchars=\\\{\}]
\prompt{bash\$} \command{export LD_LIBRARY_PATH=$LD_LIBRARY_PATH:/home/user/projects/lib}
\end{Verbatim}

Under Mac OS X, there is a similar environment variable named
\envar{DYLD\_LIBRARY\_PATH}.

\section{Summary of installation options}
\label{sec:inst-options}
%--------------------------------------------------------------------

The \verb|./configure| script accepts a large number of options. To
get the complete list of options, just execute \verb|./configure| with
the \option{-{}-help} option.

\begin{Verbatim}[commandchars=\\\{\}]
\prompt{bash\$} ./configure --help
\end{Verbatim}

Table~\ref{tab:inst-options} summarizes the most important options. 

\begin{table}[ht!]
\begin{center}
\begin{tabular}{|l|p{7cm}|}
\hline
\multicolumn{2}{|l|}{\application{libgds} build options:}\\
\hline
\hline
\option{-{}-prefix}=\parameter{PATH} & Change the installation prefix to
\parameter{PATH}. The default installation prefix is
\filename{/usr/local}.\\
\hline
\option{-{}-help} & List all the options. \\
\hline
\multicolumn{2}{}{} \\
\hline
\multicolumn{2}{|l|}{\application{C-BGP} build options:}\\
\hline
\hline

\option{-{}-prefix}=\parameter{PATH} &
Change the installation prefix to \parameter{PATH}. The default
installation prefix is \filename{/usr/local}.\\
\hline

\option{-{}-with-pcre}=\parameter{PATH} &
Specifies the location of the \application{libpcre} library and
headers. This is only required if they are not in the default library
and header search paths.\\
\hline

\option{-{}-with-jni}[=\parameter{PATH}] &
Enable the {\it Java Native Interface} (JNI). It is disabled by
default. This option can take a \parameter{PATH} value to inform about
the location of the \filename{jni.h} header.
%See Chapter~\ref{ch:jni} for more information.
\\
\hline

\option{-{}-without-jni} &
Disable the JNI.\\
\hline

\option{-{}-with-readline}[=\parameter{PATH}] &
Enable the use of the \command{libreadline} library. This library
enhances the use of \application{C-BGP} in interactive mode (see
Section~\ref{sec:user-interface-interactive}). This option can take a
\parameter{PATH} value to inform about the location of the
\application{libreadline} library and headers.\\
\hline

\option{-{}-without-readline} &
Disable the use of the \application{libreadline} library.\\
\hline

\option{-{}-disable-bgpdump} &
Disable the use of the \application{libbgpdump} library.\\
\hline

\option{-{}-disable-ipv6} &
Disable support for IPv6 in \application{libbgpdump} library.\\
\hline

\option{-{}-enable-experimental} &
Enable experimental features. Use with care.\\
\hline

\option{-{}-enable-walton} &
Enable experimental support for multiple paths (draft-walton). Use
with care.\\
\hline

\option{-{}-enable-external-best} &
Enable experimental support for external-best. Use with care.\\
\hline

\option{-{}-enable-ospf} &
Enable experimental support for OSPF. Do not use, this is broken.\\
\hline


\option{-{}-help} & List all the options. \\
\hline
\end{tabular}
\end{center}
\caption{Summary of useful \command{./configure}
  options.\label{tab:inst-options}}
\end{table}

