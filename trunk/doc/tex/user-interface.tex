%====================================================================
\chapter{User interface}
\label{ch:user-interface}
%====================================================================

%--------------------------------------------------------------------
\section{Command-line usage}
\label{sec:user-usage}
%--------------------------------------------------------------------

In order to run the \application{C-BGP} simulator, simply type 
\command{cbgp} at the command prompt\footnote{The location of the
  \command{cbgp} binary must be in your \envar{PATH} environment
  variable.}. The \command{cbgp} application supports a few
command-line parameters. These parameters are summarized in
Table~\ref{tab:cbgp-options}.

\begin{Verbatim}[commandchars=\\\{\}]
\command{cbgp} [ \option{-h} ] [ \option{-l} \textit{LOGFILE} ] [ \option{-c} \textit{SCRIPT} | \option{-e} \textit{COMMAND}] [ \option{-D} \textit{VALUE}=\textit{VALUE} ]\\
\end{Verbatim}

\begin{table}[ht!]
\begin{center}
\begin{tabular}{|l|p{7cm}|}
\hline
{\bf Option} & {\bf Description}\\
\hline
\hline
\option{-h} & Display the command-line parameters of \application{C-BGP}.\\
\hline
\option{-c} \textit{SCRIPT} & Run the simulator in {\it script
  mode}. Load and execute the specified \textit{SCRIPT} file. Note:
without any option, the simulator works in {\it interactive mode} and
commands are taken from the standard input (stdin).\\
\hline
\option{-l} \textit{LOGFILE} & Specifies the file that must be used to
record log messages. The default behaviour is to output log messages
on the standard error output (stderr).\\
\hline
\option{-e} \textit{COMMAND} & Execute a single command
\textit{COMMAND} and exits. \\
\hline
\option{-D} \textit{PARAM}=\textit{VALUE} & Set the value of parameter named
\textit{PARAM} to \textit{VALUE}. \\
\hline
\end{tabular}
\end{center}
\caption{Command-line options supported by \application{C-BGP}.\label{tab:cbgp-options}}
\end{table}

The simulator can be launched in either of two modes. The first mode
is the {\bf script mode}. To enter this mode, option \option{-c}
\textit{SCRIPT} must be specified on the command-line. In this mode,
the simulator reads a script file which contains a sequence of
commands. The commands contained in the script file are used to setup
a simulation. This mode is explained in
Section~\ref{sec:user-interface-script}.

The second mode, selected without option \parameter{-c}, is the
{\bf interactive mode}. In this mode, the simulator prompts the user
for commands. This mode is mainly intended for beginners or for
debugging purposes. This mode is also useful for designing parts of
new simulation scripts. The interactive mode is described in
Section~\ref{sec:user-interface-interactive}.

It is also possible to mix script and interactive modes by using the
\command{include} command (see \ref{cmd:include}). This command allows
to run the simulator in interactive mode, run an existing script to
setup the simulation (usually several commands), then use the
interactive command prompt to interact with the simulation state.

%--------------------------------------------------------------------
\section{Interactive mode}
\label{sec:user-interface-interactive}
%--------------------------------------------------------------------

When \application{C-BGP} is run in interactive mode, i.e. without
command-line option \option{-c}, the simulator shows a greeting
message and then prompts the user for input.

\begin{Verbatim}[commandchars=\\\{\}]
\prompt{bash\$} \command{cbgp -i}
C-BGP Routing Solver version 2.0.0-rc3

  Copyright (C) 2002-2011 Bruno Quoitin
  Networking Lab
  Computer Science Institute
  University of Mons (UMONS)
  Mons, Belgium
  
  This software is free for personal and educational uses.
  See file COPYING for details.

help is bound to '?' key
\prompt{cbgp> init.}
\prompt{cbgp>}
\end{Verbatim}

It is then possible to type in \application{C-BGP} commands. Most of
the supported \application{C-BGP} commands are described in
Chapter~\ref{ch:cmd-ref}. A simple, yet useful command is
\command{show version}. It can be used to make \application{C-BGP} report
its own version number as well as information on optional packages and
libraries. In the example below, the reported version of
\application{C-BGP} is 2.0.0-rc3. The \application{libgds} library has
the same version and \application{C-BGP} has been compiled with the
\application{libbgpdump} library. You can find more details about
command \command{show version} in section~\ref{cmd:show_version}.

\begin{Verbatim}[commandchars=\\\{\}]
\prompt{cbgp>} \command{show version}
cbgp version: 2.0.0-rc3 [bgpdump]
libgds version: 2.0.0-rc3
\prompt{cbgp>}
\end{Verbatim}

To quit \application{C-BGP} the \command{quit} command can be
used. The process will be terminated and the user will be returned to
the calling process (usually a shell interpreter). Alternatively,
pressing Ctrl-D when the command prompt is empty should also work.

\begin{Verbatim}[commandchars=\\\{\}]
\prompt{cbgp>} \command{quit}
\prompt{cbgp> done.} 
\prompt{bash\$}
\end{Verbatim}

%--------------------------------------------------------------------
\section{Command arguments}
\label{sec:user-interface-arguments}
%--------------------------------------------------------------------

This section explains how to work with commands that require
arguments. An example of such a command is \command{net add node} (see
\ref{cmd:net_add_node}). This command is used to create a new node in
the simulation. According to the command's documentation, a new node
requires an identifier. This identifier has the same form as an IPv4
address and it must be unique (i.e. not used by another node).

The new node's identifier must be provided as an argument after the
command name. The following example shows how this is done.

\begin{Verbatim}[commandchars=\\\{\}]
\prompt{cbgp>} \command{net add node 1.0.0.0}
\prompt{cbgp>}
\end{Verbatim}

If the node could be added to the simulation, no message is provided
and the simulator prompts the used for another command.

If the node could not be added, an error message is provided that
mentions the reason of the failure. For example, a node with the same
identifier could already exist. In the following example, a node with
identifier is 1.0.0.0 is added twice. An error is returned after the
second addition attempt.

\begin{Verbatim}[commandchars=\\\{\}]
\prompt{cbgp>} \command{net add node 1.0.0.0}
\prompt{cbgp>} \command{net add node 1.0.0.0}
Error: command failed
 +-- reason: could not add node (node already exists)
\prompt{cbgp>}
\end{Verbatim}

Another possible cause of error is a missing argument. In the
following example, command \command{net add node} is used without
providing a node identifier. An error is returned indicating the name
of the missing argument. In this case, it is \parameter{<addr>}

\begin{Verbatim}[commandchars=\\\{\}]
\prompt{cbgp>} \command{net add node}
Error: missing argument
 +-- input : net add node 
 +-- expect: <addr>
\prompt{cbgp>}
\end{Verbatim}

More information on the error messages returned by \application{C-BGP}
can be obtained in Section~\ref{sec:user-interface-error}.

%--------------------------------------------------------------------
\section{Command options}
\label{sec:user-interface-options}
%--------------------------------------------------------------------

This section describes how to use comands that accept optional
arguments. Let's continue with the same command as in
Section~\ref{sec:user-interface-arguments}: command \command{net add
  node}. In addition to an argument, this command also accepts several
options. We focus on a single one: option \option{-{}-no-loopback}.

The default behaviour of \command{net add node} is to add a loopback
interface to the new node. The address of this loopback interface is
the node identifier. In some cases, this behaviour is not
desirable. Option \option{-{}-no-loopback} is there for that reason:
it allows to disable the automatic creation of this loopback interface
by \command{net add node}. If the user later wants to add a loopback
interface, he must do so manually (see \ref{cmd:net_node_add_iface}).

The following example shows how to use \command{net add node} with the
\option{-{}-no-loopback} option.

\begin{Verbatim}[commandchars=\\\{\}]
\prompt{cbgp>} \command{net add node --no-loopback 1.0.0.0}
\prompt{cbgp>}
\end{Verbatim}


%--------------------------------------------------------------------
\section{Command naming}
\label{sec:user-interface-commands}
%--------------------------------------------------------------------

The names of the commands in \application{C-BGP} are organized
hierarchically. Commands related to the same simulator functions share
a common prefix. There are three main groups of commands identified
with different prefixes:

\begin{description}
\item [net]
  These are commands related to the network topology setup
  (create \hyperref[cmd:net_add_node]{nodes} and
  \hyperref[cmd:net_add_link]{links}) and the basic IP routing
  configuration (\hyperref[cmd:net_node_route_add]{add static routes},
  \hyperref[cmd:net_add_domain]{configure a simple IGP}).

\item [bgp]
  These are commands related to the BGP routing protocol
  configuration. These commands allow to
  \hyperref[cmd:bgp_add_router]{create BGP routers}, to
  \hyperref[cmd:bgp_router_add_peer]{setup BGP sessions},  to
  \hyperref[cmd:bgp_router_add_network]{advertise local networks}, 

\item [sim]
  These are commands related to the simulator. The most important
  command allows to \hyperref[cmd:sim_run]{process the pending
    events}.

\end{description}

In addition to these three groups, a group for miscellaneous commands
exists. These commands allow for example to \hyperref[cmd:print]{print
  a message} or to \hyperref[cmd:include]{include a script}. The
commands in the later group do not share a common prefix.

Within each group, subgroups were created. For example, all commands
that start with prefix \command{net node} are commands related to a
node in the network topology, all commands that start with prefix
\command{net link} are commands related to a link in the network
topology, and so on.

With this naming scheme, it should be relatively easy to figure out
what is the purpose of most commands from their names. For example,
command \command{bgp router {\it x} peer {\it y} filter out add-rule}
is a command that will add a rule to the output filter of peer {\it y}
in BGP router {\it x}.

In addition, if there is a need to enter several commands from the
same group, it is possible to first enter the command group. This way
there is no need to type the common prefix for each of the command. To
enter a group of commands, you just need to enter the prefix of the
group. For example, to enter group \command{net add}, it is possible
to enter \command{net add} in one line or to first enter \command{net}
on one line, then enter \command{add} on another line. Note on the
example how the command-line interpreter informs the user that it is
in a particular group of commands by changing the prompt.

\begin{Verbatim}[commandchars=\\\{\}]
\prompt{cbgp>} \command{net}
\prompt{cbgp-net>} \command{add}
\prompt{cbgp-net-add>} \command{node 1.0.0.0}
\prompt{cbgp-net-add>} \command{node 1.0.0.1}
\prompt{cbgp-net-add>} \command{link 1.0.0.0 1.0.0.2}
\prompt{cbgp-net-add>}
\end{Verbatim}

You can exit the current group by typing the special \command{exit}
command to go back to the previous group of commands. For example,
with the previous example, the result would be as follows. Note that
if the group had been entered in a single line \command{net add}, then
a single \command{exit} would have been needed.

\begin{Verbatim}[commandchars=\\\{\}]
\prompt{cbgp-net-add>} \command{exit}
\prompt{cbgp-net>} \command{exit}
\prompt{cbgp>} 
\end{Verbatim}

Alternatively, it is possible to exit all groups (i.e. return to the
root of the command hierarchy) by entering an empty line, as shown in
the following example.

\begin{Verbatim}[commandchars=\\\{\}]
\prompt{cbgp-net-add>}
\prompt{cbgp>} 
\end{Verbatim}

%--------------------------------------------------------------------
\section{Command context}
\label{sec:user-interface-context}
%--------------------------------------------------------------------

We have learned in Section~\ref{sec:user-interface-commands} that, by
entering a group of commands, it is possible to change the future
behaviour of the command-line interpreter in that there is no need to
type the prefix of the following commands. We say that the command
context has changed: it is now relative to a particular group of
commands.

\application{C-BGP} allows to go further by entering the context of a
particular object in the simulator: a node, a link, a BGP router, and
so on. Once the context of such an element is entered, commands
related to this element can be executed.

Let's take an example to clarify this notion. Suppose we have created
a node 1.0.0.0 as shown in
Section~\ref{sec:user-interface-arguments}. We now want to do
something with this node. For example, we want to obtain information
about its network interfaces. A first possible way to do this is to
use command \command{net node 1.0.0.0 show ifaces} as follows:

\begin{Verbatim}[commandchars=\\\{\}]
\prompt{cbgp>} \command{net node 1.0.0.0 show ifaces}
lo      1.0.0.0/32
\prompt{cbgp>}
\end{Verbatim}

Another possibility is to first enter the context of node 1.0.0.0 by
first entering \command{net node 1.0.0.0} on one line, then by
entering \command{show ifaces} on a second line. The first line enters
the context of node 1.0.0.0. It is not really a command, however it
could fail if, for example, node 1.0.0.0 did not exist.

\begin{Verbatim}[commandchars=\\\{\}]
\prompt{cbgp>} \command{net node 1.0.0.0}
\prompt{cbgp-net-node>} \command{show ifaces}
lo      1.0.0.0/32
\prompt{cbgp-net-node>}
\end{Verbatim}

To exit a context, use the \command{exit} command or an empty
line. The behaviour is exactly the same as for exiting a group of
commands.


%--------------------------------------------------------------------
\section{Auto-completion}
\label{sec:user-interface-completion}
%--------------------------------------------------------------------

This section describes the auto-completion feature in
\application{C-BGP}'s
\hyperref[sec:user-interface-interactive]{interactive mode}. The
auto-completion mechanism partially relies on the
\application{libreadline} library and will therefore only work if
\application{C-BGP} was compiled with it.

Auto-completion is a mechanism that allows to help the user complete
its current command-line input. Auto-completion can complete the name
of a command, the value of an argument, the name of an option and the
value of an option. The auto-completion feature should work for any
command name and option name. However, it is only partially supported
for arguments and options values.

To invoke the auto-completion mechanism, the user needs to use the
tabulation key once or twice. A single key press will auto-complete
the current user input if there is a single possible completion. A
second key press will provide the user with the list of possible
completions. Let's take an example to illustrate these two
situations. A tabulation key press is illustrated by \verb|[tab]|.

\begin{Verbatim}[commandchars=\\\{\}]
\prompt{cbgp>} \command{n} [tab]
\prompt{cbgp>} \command{net} [tab]
\prompt{cbgp>} \command{net} [tab]
        add
        domain
        export
        link
        links
        node
        ntf
        show
        subnet
        traffic
\prompt{cbgp>} \command{net n} [tab]
        node
        ntf
\prompt{cbgp>} \command{net no} [tab]
\prompt{cbgp>} \command{net node}
\end{Verbatim}

In the above example, the user first typed \command{n} followed by a
tabulation key press. As there is a single possible command name that
starts with \command{n}, the auto-completion is performed, and the
user input is replaced with \command{net}. Then the user again
requests auto-completion, but this time, there are multiple possible
matches (all the commands that start with \command{net}). No
auto-completion is performed. The user then presses the tabulation key
a second time to obtain the list of possible completion matches. A
list of 10 commands is provided. The user adds character 'n' to its
current input and again requests the completion list (press
\verb|[tab]| twice). Only two commands start with 'n': \command{node}
and \command{ntf}. User adds character 'o' and requests
auto-completion that can now be realized as a single possible match
remains. The user input is now \command{net node}.

To complete an argument value, the process is similar. The only
exception is that the possible completion matches might depend on the
current simulation state. Let's for example take the case where the
user wants to enter the context of one node. Assume that only two
nodes with identifiers 1.0.0.0 and 1.0.0.1 currently exist in the
simulation.

\begin{Verbatim}[commandchars=\\\{\}]
\prompt{cbgp>} \command{net node} [tab] [tab]
        1.0.0.0
        1.0.0.1
\prompt{cbgp>} \command{net node 1.0.0.1}
\end{Verbatim}

By pressing the \verb|[tab]| key twice, the user can
obtain the list of existing nodes. In this case auto-completion does
not help as the two node identifiers share a long common prefix. If a
third a node is added with identifier 1.1.0.0, then auto-completion
becomes helpful, as illustrated in the following example.

\begin{Verbatim}[commandchars=\\\{\}]
\prompt{cbgp>} \command{net node} [tab] [tab]
        1.0.0.0
        1.0.0.1
        1.1.0.0
\prompt{cbgp>} \command{net node 1.1} [tab]
\prompt{cbgp>} \command{net node 1.1.0.0}
\end{Verbatim}

%--------------------------------------------------------------------
\section{In-line help}
\label{sec:user-interface-help}
%--------------------------------------------------------------------

With the most recent versions of \application{C-BGP}
\footnote{With version 2.0.0-rc3 and later.}
, help is available within the simulator. Help files should have been
installed along with the binaries in the \verb|$prefix/share/doc|
directory where \verb|$prefix| is the base installation directory (the
default being \verb|/usr/local|). \application{C-BGP} will try to
locate these files based on the installation path.

Help can be obtained using using the special command
\command{help}. The \command{help} command takes a single argument:
the name of a command in the current context. For example, you can
obtain help about the \command{net node} command by typing
\command{help node} within the \command{net} context.

\begin{Verbatim}[commandchars=\\\{\}]
\prompt{cbgp>} \command{net}
\prompt{cbgp-net>} \command{help node}
<help should be displayed here>
\prompt{cbgp-net>}
\end{Verbatim}

To display the requested help file, the \application{less} utility is
used. You can use the up/down arrow keys to travel through the help
file. You can also perform searches through the help file using the
'/' key. Finally, you can quit the help by using the 'q' key. Please
refer to the documentation of \application{less} if you need more
information (i.e. use \command{man less}).

The following screen capture illustrates the content of the help file
for command \command{net add node}.

\begin{Verbatim}[commandchars=\\\{\},fontsize=\small,frame=single]
C-BGP Documentation              User's manual             C-BGP Documentation

\textbf{NAME}
     net add node -- add a node to the topology

\textbf{SYNOPSIS}
     node <\underline{addr}>

\textbf{ARGUMENTS}
     <\underline{addr}> address (identifier) of the new node

\textbf{DESCRIPTION}
     This command adds a new node to the topology. The node is identified by
     its IP address. This address must be unique. When created, a new node
     only supports IP routing as well as a simplified ICMP protocol. If you
     want to add support for the BGP protocol, consider using the \textbf{bgp add}
     \textbf{router} command.

     Example:


     net add node 1.0.0.1
     net add node 1.0.0.2
/usr/local/share/doc/cbgp/txt/net_add_node.txt 
\end{Verbatim}

%--------------------------------------------------------------------
\section{Command-line history}
\label{sec:user-interface-history}
%--------------------------------------------------------------------

In recent versions of \application{C-BGP}\footnote{This should be
  working since version 1.1.18. This feature requires that
  \application{C-BGP} be compiled with the \application{libreadline}
  library.}
, every time you enter a command in
interactive mode, it is registered in \application{C-BGP}'s
history. Using the up/down arrows on your keyboard, you can retrieve
past commands. The history of commands can be automatically stored in
a file and reloaded at the next execution.

The behaviour of the \application{C-BGP}'s history can be altered by
defining or re-defining two environment variables. These environment
variables control how the command-line history is stored in a
file. They are described in Table~\ref{tab:user-interface-envar}.

\begin{table}[h!]
\begin{center}
\begin{tabular}{|l|p{8cm}|}
\hline
{\bf Environment variable} & {\bf Description}\\
\hline
\hline
\envar{CBGP\_HISTFILE} &
  If the \envar{CBGP\_HISTFILE} environment variable is defined but is
  empty, \application{C-BGP} will load the command-line history from a
  file named \filename{\~{}/.cbgp\_history}. If \envar{CBGP\_HISTFILE}
  is not empty, the default file name is replaced by the environment
  variable's value.\\
\hline
\envar{CBGP\_HISTFILESIZE} &
  The \envar{CBGP\_HISTFILESIZE} environment variable can be defined
  in order to limit the number of lines that will be loaded from the
  history file. The value of \envar{CBGP\_HISTFILESIZE} must be a
  positive integer value.\\
\hline
\end{tabular}
\caption{Command-line history environment
  variables.\label{tab:user-interface-envar}}
\end{center}
\end{table}

%--------------------------------------------------------------------
\section{Script mode}
\label{sec:user-interface-script}
%--------------------------------------------------------------------

This section describes the behaviour of \application{C-BGP} when it is
run in {\bf script mode}. This mode allows to run long lists of
commands from a script file and directly obtain the result of these
commands from \application{C-BGP}'s standard output.

The same \application{C-BGP} commands can be used in interactive mode
and script mode. The behaviour of these commands will be identical in
both modes. However, in script mode, errors are not tolerated and the
processing of the script file will be terminated as soon as an error
is detected.

Usually, a script is organized as follows. The first part of the
script builds the network topology, i.e. creates nodes and links. The
second part of the script configures basic IP routing, i.e. sets up
static routes and/or configures IGP domains and IGP weights. The third
part of the script enables BGP routers, configures BGP sessions,
routing filters and originates networks. The simulation is then
run. The final part reads the results of the simulation: looks into
the routers routing tables, traces path from one router to another,
and so on.

There is however no specific organization imposed by
\application{C-BGP} on the structure of the scripts except that
commands that rely on the existence of a given object in the
simulation must no appear before the commands that create this object.

Here is a short script example that illustrates the organization
proposed in the above paragraphs:

\begin{Verbatim}[commandchars=\\\{\}]
# 1). Create topology
\command{net add node 1.0.0.0}
\command{net add node 1.0.0.1}
\command{net add link 1.0.0.0 1.0.0.1}
# 2). Setup IP routing
\command{net add domain 1 igp}
\command{net node 1.0.0.0 domain 1}
\command{net node 1.0.0.1 domain 1}
\command{net link 1.0.0.0 1.0.0.1 igp-weight --bidir 10}
\command{net domain 1 compute}
# 3). Setup BGP routing
\command{bgp add router 1 1.0.0.0}
\command{bgp router 1.0.0.0}
\command{  add network 10.0/16}
\command{  add peer 1 1.0.0.1}
\command{  peer 1.0.0.1 up}
\command{  exit}
\command{bgp add router 1 1.0.0.1}
\command{bgp router 1.0.0.1}
\command{  add peer 1 1.0.0.0}
\command{  peer 1.0.0.0 up}
\command{  exit}
# 4). Run simulation
\command{sim run}
# 5). Explore simulation results
\command{bgp router 1.0.0.1 show rib *}
\end{Verbatim}

%--------------------------------------------------------------------
\section{Error handling}
\label{sec:user-interface-error}
%--------------------------------------------------------------------

The input provided by the user might not be acceptable by the
command-line interpreter or might be erroneous. For this reason,
\application{C-BGP} will provide error messages to help the
user identify and fix the issue. This section will help to make these
messages more meaningful.

As a first example, let's take the case where an unknown command name
``foo'' is provided at the command prompt. \application{C-BGP} will
return the following error message as ``foo'' is not a supported
command:

\begin{Verbatim}[commandchars=\\\{\}]
\prompt{cbgp>} \command{foo}
Error: unknown command
 +-- input : \command{foo}
 +-- expect: bgp chdir define include net pause print require set show sim time 
\prompt{cbgp> }
\end{Verbatim}

In the above case, the error message informs that ``foo'' is an
\texttt{unknown command} in the current context. It also informs about
valid commands in this context (the \texttt{expect} part). For
example, if the user had typed ``bgp'', \application{C-BGP} would
probably have been happy with that. The error message also informs of
the location of the error. This location is shown in bold in the
\texttt{input} part. In the above example, ``foo'' is the 
offending word. But it could be different. Consider for example the
case where the user types ``bgp foo''.

\begin{Verbatim}[commandchars=\\\{\}]
\prompt{cbgp>} \command{bgp foo}
Error: unknown command
 +-- input : bgp \command{foo}
 +-- expect: add domain export link links node ntf show subnet traffic
\prompt{cbgp> }
\end{Verbatim}

In this example, the first word ``bgp'' was accepted by
\application{C-BGP} as it corresponds to the valid \command{bgp}
command, but the second word ``foo'' was rejected as ``foo'' is not a
subcommand of \command{bgp}. This is reported in the error message
where \application{C-BGP} shows that the offending word is ``foo'' in
the \texttt{input} part of the error message.

Errors reported by \application{C-BGP} can have different
causes. Table~\ref{tab:user-interface-errors} provides a summary of
the possible command-line interpreter errors.

\begin{table}
\begin{center}
\begin{tabular}{|l|p{8cm}|}
\hline
{\bf Error message} & {\bf Description}\\
\hline
\hline
{\bf unknown command} &
  The name used by the user does not correspond to a known command.\\
\hline
{\bf missing argument} &
  A command requires an argument that was not provided by the user.\\
\hline
{\bf too many arguments (vararg)} &
  A command that accepts a variable number of arguments has received
  more arguments than it can accept.\\
\hline
{\bf not a command} &
  The name used by the user is recognized by the command-line
  interpreter but it is not an executable command.\\
\hline
{\bf command failed} &
  The execution of a command failed. When this message is provided, it
  is usually accompanied by a more specific message that explains why
  the command failed.\\
\hline
{\bf bad argument value} &
  A command was provided with an invalid argument value.\\
\hline
{\bf unable to create context} &
  \\
\hline
{\bf unknown option} &
  An unknown option name was provided to a command.\\
\hline
{\bf missing option value} &
  An option was provided to a command, but no value was provided for
  this option.\\
\hline
{\bf option does not take a value} &
  An option was provided to a command along with a value but this
  option does not take a value.\\
\hline
{\bf syntax error} &
  The command-line interpreter was not able to analyse the user
  input.\\
\hline
{\bf completion could not be performed} &
  This should only occur in interactive mode. The user tried to use
  the auto-completion feature (usually by typing the <tabulation>
  several times), but no completion could be done.\\
\hline
{\bf input line too long} &
  The current input line is too long. The maximum length of an input
  line is 1024 characters.\\
\hline
%\item[command prohibited in the current context] 
{\bf error} &
  An error occurred, but it does not correspond to any of the above
  cases. Please contact the developpers.\\
\hline
{\bf unexpected error} &
  The command could not be parsed or executed due to an unexpected
  error. Please contact the developpers.\\
\hline
\end{tabular}
\caption{List of command-line interpreter errors.\label{tab:user-interface-errors}}
\end{center}
\end{table}

When an error is found in a script, the error message reports the line
where the error occurred. This behaviour is available in script mode
and in interactive mode when the
\hyperref[cmd:include]{\command{include}} command is used. The
following example illustrates this situation. A simple script file
that contains an error is executed (can you spot the error ?).

\begin{Verbatim}[commandchars=\\\{\}]
\command{net add node 1.0.0.0}
\command{net add node 1.0.0.1}
\command{net add link 1.0.0.0 1.0.0.2}
\end{Verbatim}

When the above script is executed, \application{C-BGP} reports the
error along with the line where it occurred. In this case, the error
is located in line 3. The command tries to create a link between nodes
1.0.0.0 and 1.0.0.2. The later node does not exist.

\begin{Verbatim}[commandchars=\\\{\}]
\prompt{bash\$} \command{cbgp -c script.cli}
Error: command failed
 +-- line  : 3
 +-- reason: tail-end "1.0.0.2" does not exist.
\prompt{bash\$}
\end{Verbatim}
