\begin{abstract}
C-BGP is an efficient solver for BGP, the de facto standard protocol
used for exchanging routing information accross domains in the
Internet.

C-BGP is aimed at computing the outcome of the BGP decision process in
networks composed of several routers. For this purpose, it takes into
account the routers' configuration, the externally received BGP routes
and the network topology. It supports the complete BGP decision
process, versatile import and export filters, route-reflection, and
experimental attributes such as redistribution communities. It is
easily configurable through a CISCO-like command-line interface. C-BGP
has been described in an IEEE Network magazine paper entitled
{\it "Modeling the Routing of an Autonomous System with C-BGP"} and
has been used in several research papers.

C-BGP can be used as a research tool to experiment with modified
decision processes and additional BGP route attributes. It can also be
used by the operator of an ISP network to evaluate the impact of
logical and topological changes on the routing tables computed in its
routers. Topological changes include links and routers
failures. Logical changes include changes in the configuration of the
routers such as input/output routing policies or IGP link
weights. Thanks to its efficiency, C-BGP can be used on large
topologies with sizes of the same order of magnitude than the
Internet.

\vspace{2cm}
\noindent Published: 2011\\
$\copyright$ 2003-2011, Bruno Quoitin\\
Networking Lab\\
Computer Science Department\\
Science Faculty\\
University of Mons\\
Belgium

\vspace{2cm}
\noindent
The authors of C-BGP are grateful for the numerous comments and
advices provided by the following persons: Cristel Pelsser (IIJ, Japan), Steve
Uhlig (Deutsche Telekom, Germany), Olivier Bonaventure (UCLouvain,
Belgium), Olivier Delcourt, Simon
Balon, Jean Lepropre, Gael Monfort, Fabian Skiv�e and Guy Leduc (ULg,
Belgium), Wolfgang M�hlbauer (ETH, Zurich), Olaf Maennel
(Loughborough, UK) and Anja Feldmann (Univ. Berlin / Deutsche Telekom,
Germany), Gregory Culpin, Chris Erway (Brown University, USA), Ricardo
Oliveira (UCLA, USA), Wang Lijun (Tsinghua University, China), Joseph
Emeras (LIRMM, France), Marco Ruffini (University of Dublin, Ireland),
Didier Bousser (WANDL), Tomasz Szewczyk (PSNC, Poland), Tim Griffin
(Univ. of Cambridge), Stefano Secci (ENST, Paris), Selin
Cerav-Erbas, Jean-Fran�ois Paque, Nicolas Fournier, Tarek Guerniche,
Nick Feamster (Georgia Tech, USA), Amir H Rasti (U. of Oregon, USA),
Arnaud Coomans (UCL, Belgium), Adhy S. Bramantyo (ITB, Indonesia),
Andrea di Menna and Luca Cittadini (Universita Roma Tre, Italy),
J�rgen Sch�nw�lder (Jacobs University, Bremen, Germany), Thomas
Telkamp (Cariden), Jong Han Park (UCLA), Laurent Vanbever (UCLouvain),
Pascal Merindol (Univ. of Strasbourg). The authors would also like
to thank Vincent Letocart for early discussions about C
programming and to Dan Ardelean for sharing his libbgpdump library.

\vspace{2cm}
\noindent
This work was supported by the European Commission within the ATRIUM
and AGAVE projects and by the Walloon Region under the TOTEM and
SPINNET projects. This work also received support from the e-NEXT
European Network of Excellence and by a grant from France Telecom
R\&D.
\end{abstract}
